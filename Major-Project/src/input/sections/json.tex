\section{JSON}

JSON (JavaScript Object Notation) is a lightweight data-interchange format. It is easy for humans to read and write. It is easy for machines to parse and generate. JSON is a text format that is completely language independent but uses conventions that are familiar to programmers of the C-family of languages, including C, C++, C\#, Java, JavaScript, Perl, Python, and many others. These properties make JSON an ideal data-interchange language.

\begin{figure}
	\centering
	\includegraphics[width=0.4\linewidth]{images/JSON_vector_logo}
	\caption{JSON Logo}
	\label{fig:240px-jsonvectorlogo}
\end{figure}

JSON is an open, text-based data exchange format. Like XML, it is human-readable, platform independent, and enjoys a wide availability of implementations. Data formatted according to the JSON standard is lightweight and can be parsed by JavaScript implementations with incredible ease, making it an ideal data exchange format for Ajax web applications. Since it is primarily a data format, JSON is not limited to just Ajax web applications, and can be used in virtually any scenario where applications need to exchange or store structured information as text.

JSON is built on two structures:

\begin{itemize}
	\item A collection of name/value pairs. In various languages, this is realized as an object, record, struct, dictionary, hash table, keyed list, or associative array.
	\item An ordered list of values. In most languages, this is realized as an array, vector, list, or sequence.
\end{itemize}

These are universal data structures. Virtually all modern programming languages support them in one form or another. It makes sense that a data format that is interchangeable with programming languages also be based on these structures.
