\section{Boost Libraries}
\begin{figure}[H]
    \centering \includegraphics[width=0.7\linewidth]{images/boost.png}
    \caption{The Boost Libraries}
\end{figure}
Boost is a set of libraries for the C++ programming language that provide support for tasks and structures such as linear algebra, pseudorandom number generation, multithreading, image processing, regular expressions, and unit testing. It contains over eighty individual libraries.\\
Most of the Boost libraries are licensed under the Boost Software License, designed to allow Boost to be used with both free and proprietary software projects. Many of Boost's founders are on the C++ standards committee, and several Boost libraries have been accepted for incorporation into both the C++ Technical Report 1 and the C++11 standard.\\
The libraries are aimed at a wide range of C++ users and application domains. They range from general-purpose libraries like the smart pointer library, to operating system abstractions like Boost FileSystem, to libraries primarily aimed at other library developers and advanced C++ users, like the template metaprogramming (MPL) and domain-specific language (DSL) creation (Proto).\\
In order to ensure efficiency and flexibility, Boost makes extensive use of templates. Boost has been a source of extensive work and research into generic programming and metaprogramming in C++.\\
Most Boost libraries are header based, consisting of inline functions and templates, and as such do not need to be built in advance of their use. Some Boost libraries coexist as independent libraries.\\
The original founders of Boost still active in the community include Beman Dawes and David Abrahams. Author of several books on C++, Nicolai Josuttis contributed to the Boost array library in 2001. There are mailing lists devoted to Boost library use and library development, active as of 2015.
