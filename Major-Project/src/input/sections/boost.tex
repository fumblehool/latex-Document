\section{Boost Libraries}
\begin{figure}[H]
    \centering \includegraphics[width=0.7\linewidth]{images/boost.png}
    \caption{The Boost Libraries}
\end{figure}
Boost is a set of libraries for the C++ programming language that provide support for tasks and structures such as linear algebra, pseudorandom number generation, multithreading, image processing, regular expressions, and unit testing. It contains over eighty individual libraries.\\
Most of the Boost libraries are licensed under the Boost Software License, designed to allow Boost to be used with both free and proprietary software projects. Many of Boost's founders are on the C++ standards committee, and several Boost libraries have been accepted for incorporation into both the C++ Technical Report 1 and the C++11 standard.\\
The libraries are aimed at a wide range of C++ users and application domains. They range from general-purpose libraries like the smart pointer library, to operating system abstractions like Boost FileSystem, to libraries primarily aimed at other library developers and advanced C++ users, like the template metaprogramming (MPL) and domain-specific language (DSL) creation (Proto).\\
In order to ensure efficiency and flexibility, Boost makes extensive use of templates. Boost has been a source of extensive work and research into generic programming and metaprogramming in C++.\\
The Boost community is responsible for developing and publishing the Boost libraries. The community consists of a relatively large group of C++ developers from around the world coordinated through the web site www.boost.org as well as several mailing lists. GitHub is used as the code repository. The mission statement of the community is to develop and collect high-quality libraries that complement the standard library. Libraries that prove of value and become important for the development of C++ applications stand a good chance of being included in the standard library at some point.

The Boost community emerged around 1998, when the first version of the standard was released. It has grown continuously since then and now plays a big role in the standardization of C++. Even though there is no formal relationship between the Boost community and the standardization committee, some of the developers are active in both groups. The current version of the C++ standard, which was approved in 2011, includes libraries that have their roots in the Boost community.

The Boost libraries are a good choice to increase productivity in C++ projects when your requirements go beyond what is available in the standard library. Because the Boost libraries evolve faster than the standard library, you have earlier access to new developments, and you don’t need to wait until those developments have been added to a new version of the standard library. Thus, you can benefit from progress made in the evolution of C++ faster, thanks to the Boost libraries.

Due to the excellent reputation of the Boost libraries, knowing them well can be a valuable skill for engineers. It is not unusual to be asked about the Boost libraries in an interview because developers who know these libraries are usually also familiar with the latest innovations in C++ and are able to write and understand modern C++ code.
Most Boost libraries are header based, consisting of inline functions and templates, and as such do not need to be built in advance of their use. Some Boost libraries coexist as independent libraries.\\
The original founders of Boost still active in the community include Beman Dawes and David Abrahams. Author of several books on C++, Nicolai Josuttis contributed to the Boost array library in 2001. There are mailing lists devoted to Boost library use and library development, active as of 2015.
