\section{Bison}

\begin{figure}[h]
	\centering \includegraphics[scale=0.2]{images/gnu.png}
	\caption{GNU Logo}
	\label{fig:gnu}
\end{figure}

Bison is a general-purpose parser generator that converts an annotated context-free grammar into a deterministic LR or generalized LR (GLR) parser employing LALR(1) parser
tables.
Once you are proficient with Bison, you can use it to develop a wide range of language parsers, from those used in simple desk calculators to complex programming languages.


Bison is upward compatible with Yacc: all properly-written Yacc grammars ought to work with Bison with no change. Anyone familiar with Yacc should be able to use Bison with little trouble. You need to be fluent in C or C++ programming in order to use Bison
or to understand this manual. Java is also supported as an experimental feature.


Bison was written originally by Robert Corbett. Richard Stallman made it Yacc-compatible. Wilfred Hansen of Carnegie Mellon University added multi-character string literals and other features. Since then, Bison has grown more robust and evolved many
other new features thanks to the hard work of a long list of volunteers.

Bison .y specification file:

\begin{verbatim}
	/*** Definition section ***/
	
	%%
	
	/*** Rules section ***/
	
	%%
	
	/*** C Code section ***/

\end{verbatim}
