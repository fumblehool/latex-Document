\section{C++}

\begin{figure}[H]
	\centering \includegraphics[scale=0.3]{images/c++.png}
	\caption{C++ Logo}
	\label{fig:c++}
\end{figure}

C++ is a general-purpose programming language. It has imperative, object-oriented and generic programming features, while also providing facilities for low-level memory manipulation.

It was designed with a bias toward system programming and embedded, resource-constrained and large systems, with performance, efficiency and flexibility of use as its design highlights. C++ has also been found useful in many other contexts, with key strengths being software infrastructure and resource-constrained applications, including desktop applications, servers (e.g. e-commerce, web search or SQL servers), and performance-critical applications (e.g. telephone switches or space probes). C++ is a compiled language, with implementations of it available on many platforms and provided by various organizations, including the Free Software Foundation (FSF's GCC), LLVM, Microsoft, Intel and IBM.

C++ is standardized by the International Organization for Standardization (ISO), with the latest standard version ratified and published by ISO in December 2014 as ISO/IEC 14882:2014 (informally known as C++14). The C++ programming language was initially standardized in 1998 as ISO/IEC 14882:1998, which was then amended by the C++03, ISO/IEC 14882:2003, standard. The current C++14 standard supersedes these and C++11, with new features and an enlarged standard library. Before the initial standardization in 1998, C++ was developed by Bjarne Stroustrup at Bell Labs since 1979, as an extension of the C language as he wanted an efficient and flexible language similar to C, which also provided high-level features for program organization.

In 1979, Bjarne Stroustrup, a Danish computer scientist, began work on "C with Classes", the predecessor to C++. The motivation for creating a new language originated from Stroustrup's experience in programming for his Ph.D. thesis. Stroustrup found that Simula had features that were very helpful for large software development, but the language was too slow for practical use, while BCPL was fast but too low-level to be suitable for large software development. When Stroustrup started working in AT\&T Bell Labs, he had the problem of analyzing the UNIX kernel with respect to distributed computing. Remembering his Ph.D. experience, Stroustrup set out to enhance the C language with Simula-like features. C was chosen because it was general-purpose, fast, portable and widely used. As well as C and Simula's influences, other languages also influenced C++, including ALGOL 68, Ada, CLU and ML.

Many other programming languages have been influenced by C++, including C\#, D, Java, and newer versions of C (after 1998).

Features of Language:
\begin{enumerate}
	\item Object Storage.
	\begin{enumerate}
		\item Static storage duration objects.
		\item Thread storage duration objects.
		\item Automatic storage duration objects.
		\item Dynamic storage duration objects.
	\end{enumerate}
	\item Templates
	\item Objects.
	\begin{enumerate}
		\item Encapsulation.
		\item Inheritance.
	\end{enumerate}
	\item Operators and operator overloading.
	\item Polymorphism.
	\begin{enumerate}
		\item Static polymorphism
		\item Dynamic polymorphism
		\begin{enumerate}
			\item Inheritance
			\item Virtual member functions
		\end{enumerate}
	\end{enumerate}
\item Lambda expressions
\item Exception Handling
\end{enumerate}
