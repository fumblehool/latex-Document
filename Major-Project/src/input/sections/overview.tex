\section{Overview}
\begin{figure}[H] 
	\centering \includegraphics[scale=0.31]{images/openscad.png}
	\caption{OpenSCAD's logo}
	\label{fig:openscadlogo}
\end{figure}
Multi-threaded compile and render for OpenSCAD is our major project. It is under the umbrella organization of BRL-CAD. OpenSCAD is a free and Open-source software application for creating solid 3D CAD objects. It is a script only based modeler, with a specific description language. Parts cannot be selected or modified by mouse in the 3D view. An OpenSCAD script specifies geometric primitives and defines how they are modified and manipulated to render a 3D model. OpenSCAD is available for Windows, Linux and OS X. It does constructive solid geometry (CSG).

OpenSCAD has in a way redefined how easy 3D modeling can be. But the Wikipedia article on OpenSCAD says that it is a non-interactive modeler, but rather a 3D compiler based on a textual description language. Pay attention to the above line, it is primarily what I will be talking about.

Solid 3D modeling. That sounds like some serious business. But it's just an awesome tool for making models pertaining to many uses (mostly 3D printing). And 3D printing as we can all agree upon is cool. 3D models can be created by anyone using OpenSCAD. OpenSCAD is as much for designers as it is for you and me. What else can most people agree upon apart from the fact that solid 3D modeling is cool? A graphical interface is simpler and more intuitive to use. There is a general aversion for typing commands in order to get things done. Simply put, more people have an inclination towards GUI. Another concern is speed at which the results are presented before the user. It would be a waste of CPU capacity if not all of it is being used in the compilation and rendering of the models created in OpenSCAD. This can be achieved by spliting the process into multiple threads.

Since the rendering of various models internally done using the CGAL library, it becomes very important to check how the library behaves on a multi-threaded approach. This is something that will be considered before actually starting the implementation.

