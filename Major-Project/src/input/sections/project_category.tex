\section{Project Category}
OpenSCAD is a free and Open-source software application for creating solid 3D CAD objects. It is under the umbrella organisation of BRL-CAD. Since our project comes under OpenSCAD, it can be classified as being an application development project. But as much as it is an application development project, it is also a research project. It took a fair share of observation and study in order to find the right approach the problem. The majority of the time was spent in figuring out how the internals of this huge piece of software work. The study involved learning about various mathematical constructs used in the geometry of the models. Researching the computer graphics part of the sotware was also very essential in order to form a stronger understanding of how the problem in hand is to be tackled. As mentioned before, the CGAL library (which is used to render the models) is not entirely thread safe. So it bacame very crucial to analyse what parts of it are being used in the render process and how these parts actually work.
Upon researching the above mentioned things, the right way was to be decided upon. The project essentially changes the design approach of an important segament of OpenSCAD i.e. rendering. So of course it required great consideration before implementing it.
