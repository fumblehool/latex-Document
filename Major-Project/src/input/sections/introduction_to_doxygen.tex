\section{Introduction to Doxygen}
\begin{figure}[H]
\centering \includegraphics[scale=1.5]{images/Doxygen.png}
\caption{Doxygen Logo}
\end{figure}
Doxygen is a documentation generator, a tool for writing software reference 
documentation. The documentation is written within code, and is thus 
relatively easy to keep up to date. Doxygen can cross reference 
documentation and code, so that the reader of a document can easily 
refer to the actual code.

Doxygen supports multiple programming languages, especially C++, C, 
C\#, Objective-C, Java, Python, IDL, VHDL, Fortran and PHP. Doxygen
 is free software, released under the terms of the GNU General Public 
License.\\

\subsection{Features of Doxygen}
\begin{itemize}
\item Requires very little overhead from the writer of the documentation. 
Plain text will do, Markdown is support, and for more fancy or structured 
output HTML tags and/or some of doxygen's special commands can be used.
\item Cross platform: Works on Windows and many Unix flavors (including 
Linux and Mac OS X).
\item Comes with a GUI frontend (Doxywizard) to ease editing the options 
and run doxygen. The GUI is available on Windows, Linux, and Mac OS X.
\item Automatically generates class and collaboration diagrams in HTML 
(as clickable image maps) and $\mbox{\LaTeX}$ (as Encapsulated PostScript 
images).
\item Allows grouping of entities in modules and creating a hierarchy 
of modules.
\item Doxygen can generate a layout which you can use and edit to change 
the layout of each page.
\item Can cope with large projects easily.
\end{itemize}

\begin{figure}
\centering \includegraphics[scale=.34]{images/doc.png}
\caption{Documentation using Doxygen (main page)}
\end{figure}


\begin{figure}
\centering \includegraphics[scale=.34]{images/doc1.png}
\caption{Doxygen documentation of a function}
\end{figure}
\begin{figure}
\centering \includegraphics[scale=.34]{images/doc2.png}
\caption{Documentation using Doxygen(list of files)}
\end{figure}


