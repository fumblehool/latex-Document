\section{Objective of Project}

One of the primary benefits of OpenSCAD is the ability to design customizable content. These are designs which are parametrized using parameters or top-level variables.


Some projects utilize OpenSCAD's ability to customize designs as part of their web services.
e.g.
\begin{enumerate}
    \item Thingiverse Customizer
    \item  Sculpteo Parametric Designs
    \item e-NABLE Handomatic.
\end{enumerate}


The goal of this project is two-fold:
\begin{enumerate}
    \item  Offer an auto-generated GUI associated with a customizable design, making it easier to both create and use such designs
    \item Offer an authoritative standard for how to specify meta-data to guide the generation of such a GUI.
   
\end{enumerate}

As a temporary measure, we're also planning to support the meta-data syntax used by Thingiverse, making it possible to use the thousands of customizable designs published there.


%The major Objectives of this project are:
%\begin{enumerate}
%    \item Syntax support for generation of customization form:
%    The customization form generated on Thingiverse is based on a certain syntax for both describing the elements in the form and providing a range of their values. In order to make this work in OpenSCAD as well, the same style of description and parameterization can be incorporated into OpenSCAD. Hence the user will be able to generate the customization form from within OpenSCAD by adding a few simple lines in the .scad file.
%  
%    Customization of the model from the form:
%    Once the form is ready, it must be able to customize the model as desired by the user. The changes made in the form should directly correspond to changes in the model itself.
%    Enhancing the UI for the customization form:
%    The customization form is there to make the whole customization thing easy. And that implies that the form itself should also be easy to use. And this can be achieved by having a good and simple look to the whole thing.
%  
%\end{enumerate}

