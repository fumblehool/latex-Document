\section{User Requirement Analysis}
User Requirements Analysis for a software system is a complete description of the requirements of the User. It includes functional Requirements
and Non-functional Requirements. Non-functional requirements are
requirements which impose constraints on the design or implementation.

 
{\bf Users of the System:}
    \begin{enumerate}
        \item Modeler: Modelers are the people how the code to create model theirs for their own use or for commercial use.
        \item Client: These are the people which will customize model to their need made by modelers before ordering or printing model themselves.
        \item Other Softwares: Many web Apps or other software using OpenSCAD as there Backend might need to interact with the customizer.
               
    \end{enumerate}

\subsection{Functional Requirements}
\begin{itemize}
    \item {\bf Syntax to generate a widget to modify parameter}: This means there should be a syntax which could be used to generate a different widget for a different type of parameters.
    The widgets that we intend to support are:
    \begin{enumerate}
        \item Spinbox
            \begin{enumerate}
                \item With Increment Size
                \item Without Increment Size
            \end{enumerate}
        \item Checkbox
        \item Slider
            \begin{enumerate}
                \item With Increment Size
                \item With Default Increment Size
            \end{enumerate}
        \item Textbox
        \item VectorWidget
        \item Combo Box
            \begin{enumerate}
                \item Simple
                \item Labeled
            \end{enumerate}
   
    \begin{table}[h]
        \centering
        \caption{Table to show the required support of widgets for different DataType}
        \begin{tabular}{ |c|c|c|c|c|c| }
            \hline
            & \multicolumn{5}{|c|}{Type of Widgets} \\
            \hline
            Type of Data&    Number&    String&    BOOL &Vector &Range     \\ [0.5ex]
            \hline
            SpinBox&Y&    -&    -&    -&    - \\ \hline
            ComboBox&    Y&    Y&    -&    -&    - \\ \hline
            Text&    Y&    Y&    -&    O&    - \\ \hline
            Slider&    Y&    -&    -&    -&    - \\ \hline
            VectorWidget&    -&    -&    -&    Y&    - \\ \hline
            Checkbox&    -&    -&    Y&    -&    - \\ [1ex]
            \hline
        \end{tabular}
        \label{table2}
    \end{table}
   
    \end{enumerate}
    \item {\bf Syntax to Describe parameter}
        This means there should be a syntax which could be used to provide a description for the parameter.
    \item \textbf{Syntax to Group Parameter}
        This means there should be a syntax which could be used to group the parameters into different groups or tab to easily manage Customizer and make customizer interface little simple.
    \item \textbf{Syntax to Hide parameters}
        This means there should be a syntax which could be used to Hiding certain parameters.
    \item \textbf{Syntax to make certain parameters Global}
        This means there should be a syntax which could be used to make certain parameters global i.e. they are present in each and every group.
    \item \textbf{Save the set of parameters in JSON file}
    This feature means there should be a way a way that gives user the ability to save the value of the parameter and also we can apply them through the cmd-line and get the output.
	
	\item\textbf{Provides option to reset parameters} All parameters in customizer could be reset to
	default value just by the click of a button.
    \item \textbf{GUI to add Set of Parameters}
        This means there should be a way to store different set of parameters which represent different models from generic model.
  \item \textbf{ Cmd-line support to apply Set of Parameters} This means that values of
    the parameter for given set can be applied to model using the cmd-line argument.
    \end{itemize}
\subsection{Non-functional requirements}
\begin{enumerate}
    \item Extensible: It should be able to support future functional requirements
    \item Usability: Simple user interfaces that a layman can understand.
    \item Modular Structure: The software should have  structure. So, that different parts of software would be changed without affecting other parts.
    \item Backward compatibility: Addition of new syntax should not forbid script to work correctly on the backward versions of OpenSCAD.
\end{enumerate}

