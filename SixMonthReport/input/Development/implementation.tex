\section{Implementation}
\subsection{Customizer}

\begin{figure}
    \centering \includegraphics[scale=0.60]{images/output/1.png}
    \caption{StartUp Screen for OpenSCAD}
    \label{fig:1}
\end{figure}

Customizer  will provide User Interface to Customize Models interactively instead of modifying them manually. It will make the user able to create the templates for given model which can further be customized to cater to their need of different users and also provide a feature to save the set of parameters which define a different model using the same template of the model.
\subsection{Activation of Customizer functions}
\begin{itemize}
   
    \item This is experimental functionality.So, Initially
    OpenSCAD will look like \ref{fig:Normal OpenSCAD}
    \item In [Edit] menu, select [Preferencews] then open tab [Features], tick Customizer, then close the window when tick shown \ref{fig:3}.
    \item In View menu, you shall now have an option [Hide customizer], that you shall untick. Then you will be able to see the customizer \ref{fig:OpenSCAD with Customizer}
   
\end{itemize}

\begin{figure}
       \centering \includegraphics[width=\linewidth]{images/output/2.png}
       \caption{OpenSCAD without customizer}
       \label{fig:Normal OpenSCAD}
\end{figure}
\begin{figure}
       \centering \includegraphics[width=\linewidth]{images/output/3.png}
       \caption{Preferences Widget to activate Customizer}
       \label{fig:3}
\end{figure}
\begin{figure}
       \centering \includegraphics[width=\linewidth]{images/output/5.png}
       \caption{OpenSCAD with Customizer }
       \label{fig:OpenSCAD with Customizer}
\end{figure}

\subsection{Syntax support for generation of the customization form}

    \begin{lstlisting}[language=c++]
    // variable description
    variable name = defaultValue; // possible values
    \end{lstlisting}

Following is the syntax for how to define different types of widgets in the form

\begin{enumerate}
    \item \textbf{Drop down box}
    \begin{lstlisting}[language=c++]
    // combo box for nunber
    Numbers=2; // [0, 1, 2, 3]
   
    // combo box for string
    Strings="foo"; // [foo, bar, baz]
   
    //labeled combo box for numbers
    Labeled_values=10; // [10:L, 20:M, 30:L]
   
    //labeled combo box for string
    Labeled_value="S"; // [S:Small, M:Medium, L:Large]
    \end{lstlisting}
    \item \textbf{Slider} Only numbers are allowed in this one, specify any of the following:
    \begin{lstlisting}[language=c++]
    // slider widget for number
    slider =34; // [10:100]
   
    //step slider for number
    stepSlider=2; //[0:5:100]
    \end{lstlisting}
    \item \textbf{Checkbox}
    \begin{lstlisting}[language=c++]
    //description
    Variable = true;
    \end{lstlisting}
    \item \textbf{Spinbox}
    \begin{lstlisting}[language=c++]
    // spinbox with step size 1
    Spinbox= 5;
    \end{lstlisting}
    \item \textbf{Textbox}
    \begin{lstlisting}[language=c++]
    //Text box for vector with more than 4 elements
    Vector=[12,34,44,43,23,23];
   
    // Text box for string
    String="hello";
   
    \end{lstlisting}
    \item \textbf{Special vector}
    \begin{lstlisting}[language=c++]
   
    //Text box for vector with less than or equal to 4 elements
    Vector2=[12,34,45,23];
    \end{lstlisting}

\end{enumerate}

\subsection{Creating Tabs}
Parameters can be grouped into \textbf{tabs}. This feature will allow us to separate similar and related parameters. The syntax for this is also mainly similar to that of Thingiverse syntax for creating the tabs. To create a tab, use a multi-line block comment like this:

\textbf{/* [Tab Name] */}


Screenshot number \ref{fig:5} Shows the implemenation of this feature.

The following tab names are reserved for special functionality:
\begin{description}
\item [Global] Parameters in the global tab will always be shown on every tab no matter which tab is selected. Note: there will be no tab for “Global” params, they will just always be shown in all the tabs.

\item [Hidden] Parameters in the hidden tab will never be displayed. Not even the tab will be shown. Even though the variables who have not been parameterized using the Thingiverse or native syntax will not be displayed in OpenSCAD parameter widget but we have implemented this to make our comment like syntax similar as that of Thingiverse.
\end{description}

Also, the parameters who are under no tab will be displayed under TAB named “parameters”.
\begin{lstlisting}[language=c++]
    // combo box for nunber
    Numbers=2; // [0, 1, 2, 3]
   
    // combo box for string
    Strings="foo"; // [foo, bar, baz]
   
   
    /*[ Slider ]*/
    // slider widget for number
    slider =34; // [10:100]
   
    //step slider for number
    stepSlider=2; //[0:5:100]
   
    /* [Global] */
   
    //description
    Variable = true;
   
    /*[Hidden] */
   
    // spinbox with step size 1
    Spinbox = 5;
   
    /* [Textbox] */
   
    //Text box for vector with more than 4 elements
    Vector=[12,34,44,43,23,23];
   
    // Text box for string
    String="hello";
   
\end{lstlisting}

\begin{figure}
    \centering \includegraphics[width=\linewidth]{images/output/6.png}
    \caption{Shows different groups generated through customzier}
    \label{fig:5}
\end{figure}

\subsection{ Saving Parameters value in JSON file}
This feature which is unique to openSCAD give the user the ability to save the values of all parameters and also we can apply them through the cmd-line and get the output.

And JSON file is written in the following format:

\begin{lstlisting}[language=Java]
{
    "parameterSets":
    {
            "fileFormatVersion": "1"
        "set-name":
        {
            "parameter-name" :"value",
            "parameter-name" :"value"
        },
        "set-name":{
            "parameter-name" :"value",
            "parameter-name" :"value"
        },
    },
    "fileFormatVersion": "1"
}
\end{lstlisting}

\textbf{Example:}
\begin{lstlisting}[language=Java]
{
    "parameterSets":
    {
        "FirstSet":
        {
            "Labled_values": "13",
            "Numbers": "18",
            "Spinbox": "35",
            "Vector": "[2,34,45,12,23,56]",
            "slider": "2",
            "stepSlider": "12",
            "string": "he"
        },
        "SeconSet":
        {
            "Labled_values": "10",
            "Numbers": "8",
            "Spinbox": "5",
            "Vector": "[12,34,45,12,23,56]",
            "slider": "12",
            "stepSlider": "2",
            "string": "hello"
        }
    },
    "fileFormatVersion": "1"
}
\end{lstlisting}

You can write the JSON file using the two methods:
\begin{itemize}
    \item Manually writing the JSON file
    \item Using the OpenSCAD's Customizer GUI
\end{itemize}

\subsection{Appling Parameters sets from JSON file}
To select the parameter set from the JSON file and apply them on the model. We have two options:

\begin{itemize}
    \item Cmdline
    \item GUI
\end{itemize}

\subsubsection{Cmdline}
Cmdline option allows use for apply differenet set of parameters to the model without using the GUI and it also helps to use all the features provided by the customzier to be accessed using the cmdline. This feature will help other web-based softwares to utilize this feature. You can see various options available though cmdline in the figure \ref{fig:7}
 
\begin{lstlisting}[language=bash]
openscad --enable=customizer -o model-2.stl -p parameters.json -P 
model-2 model.scad
\end{lstlisting}

\begin{lstlisting}[language=bash]
openscad --enable=customizer -o <output-file> -p <parameteric-file> -P 
<NameOfSet> <input-file SCAD file >\end{lstlisting}

\begin{itemize}
    \item -p is used to give input JSON file in which parameters are saved.
    \item -P is used to give the name of the set of the parameters written in JSON file.
\end{itemize}

\begin{figure}
    \centering \includegraphics[width=\linewidth]{images/output/8.png}
    \caption{Cmdline options available for customizer}
    \label{fig:7}
\end{figure}

\subsubsection{GUI}

Through GUI you can easily apply and save Parameter in JSON file using Present section in Customizer explained below.

In customizer, You will be able to see two checkbox’s which are
\begin{itemize}
\item \textbf{Automatic Preview:}
If checked preview of the model will be automatically updated when you change any parameter in Customizer else you need to click preview button after you update parameter in the customizer.
\item \textbf{Show Details:}
If checked the description for the parameter will be shown above the input widget for the parameter else It will not be displayed but you still can view the description by hovering the cursor over the input widget.
\end{itemize}
Then comes \textbf{Reset} button which when clicked resets the values of all input widgets for the parameter to default provided in SCAD file.

Next, come Preset section: It consist of three buttons
\begin{description}
    \item \textbf{Combo Box:}
        It is used to select the set of parameters to be used
    \item \textbf{+ button:}
    \begin{enumerate}
        \item It is used to update the set selected in combo Box. On clicking + button values of parameters in set are replaced by new values
        \item If we select “No set selected” in comboBox, then we can use + button to add new set of the parameters \ref{fig:Add new set}
    \end{enumerate}
    \begin{figure}
        \centering \includegraphics[width=\linewidth]{images/output/7.png}
        \caption{Widget to add new set to JSON file}
        \label{fig:Add new set}
    \end{figure}

    \item \textbf{$-$ button: }
        It is used to delete the set selected in combo Box.
    and finally below Preset Section is the Place where you can play with the parameters.
\end{description}


You can also refer to  two examples that are Part of OpenSCAD to learn more
\begin{enumerate}
    \item Parametric/sign.scad
    \item Parametric/candlStand.scad
\end{enumerate}

After running a set of commands of \LaTeX and SageMath, it produces the
output in PDF form (with pdflatex) Figure \ref{fig:5}.

\begin{figure}
\centering \includegraphics[width=\linewidth]{images/output/9.png}
\caption{customizer with different set of parameters selected}
\label{fig:8}
\end{figure}