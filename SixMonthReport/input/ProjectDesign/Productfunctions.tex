\section{Product Functions}

Functions performed by Customizer are:
\begin{itemize}
    \item {\bf Provides Syntax to generate a widget to modify parameter}: This means there is a syntax which could be used to generate a different widget for a different type of parameters.
    The widgets that we intend to support are:
    \begin{enumerate}
        \item Spinbox
        \begin{enumerate}
            \item With Increment Size
            \item Without Increment Size
        \end{enumerate}
        \item Checkbox
        \item Slider
        \begin{enumerate}
            \item With Increment Size
            \item With Default Increment Size
        \end{enumerate}
        \item Textbox
        \item VectorWidget
        \item Combo Box
        \begin{enumerate}
            \item Simple
            \item Labeled
        \end{enumerate}
       
        \begin{table}[h]
            \centering
            \caption{Table to show the support of widgets for different DataType}
            \begin{tabular}{ |c|c|c|c|c|c| }
                \hline
                & \multicolumn{5}{|c|}{Type of Widgets} \\
                \hline
                Type of Data&    Number&    String&    BOOL &Vector &Range     \\ [0.5ex]
                \hline
                SpinBox&Y&    -&    -&    -&    - \\ \hline
                ComboBox&    Y&    Y&    -&    -&    - \\ \hline
                Text&    Y&    Y&    -&    Y&    - \\ \hline
                Slider&    Y&    -&    -&    -&    - \\ \hline
                VectorWidget&    -&    -&    -&    Y&- \\ \hline
                Checkbox&    -&    -&    Y&    -&    - \\ [1ex]
                \hline
            \end{tabular}

            \label{table2}
        \end{table}
       
    \end{enumerate}
   
    \item \textbf{Provide Syntax for attribute of parameter widget to be created:}
    Customer also provide syntax to provide additional attribute of the parameter
   
    \begin{table}[h]
        \centering
        \caption{Table to show the support of attributes  for different widgets}
        \begin{tabular}{ |c|c|c|c|c|c|c| }
            \hline
            & \multicolumn{6}{|c|}{Type of Widgets} \\
            \hline
            &Max&    Min &    List of items&    Step size&    Label List     &Default     \\ [0.5ex]
            \hline
            SpinBox& -&    -&    -&    Y&     -& Y \\ \hline
            ComboBox&    -&    -&    Y&    -&    Y&Y \\ \hline
            Text&    -&    -&    -&    -&    -&Y \\ \hline
            Slider&    Y&    Y&    -&    Y&    -&Y \\ \hline
            Vector&    -&    -&    -&    -&    -&Y \\ \hline
            Checkbox&    -&    -&    -&    -&    -&Y \\ [1ex]
            \hline
        \end{tabular}
        \label{table2}
    \end{table}
   
    \item {\bf Provides Syntax to Describe parameter}:
    This means there is a syntax which could be used to provide a description for the parameter.
   
    \item \textbf{Provides Syntax to Group Parameter}:
    This means there is a syntax which could be used to group the parameters into different groups or tab to easily manage Customizer and make customizer interface little simple.
   
    \item \textbf{Provides Syntax to Hide parameters}
    This means there is a syntax which could be used to Hiding certain parameters.
   
    \item \textbf{Provides Syntax to make certain parameters Global}
    This means there is a syntax which could be used to make certain parameters global i.e. they are present in each and every group.

    \item \textbf{Provides option to reset parameters:}
    All parameters in customizer could be reset to default value just by the click of a button.

    \item \textbf{Provides option to Hiding description:}
    Description of all the parameters in customizer could be Hidden value just by the click of a button.
   
    \item \textbf{Provides Save the set of parameters in JSON file}:
    This feature provides a way that gives user the ability to save the value of the parameter and also we can apply them through the cmd-line and get the output.
   
    \item \textbf{Provides GUI to add Set of Parameters}
    This means there is a way to store a different set of parameters which represent different models from generic model.
   
    \item \textbf{Provides Cmd-line support to apply Set of Parameters:}
    This means that values of the parameter for given set can be applied to model using the cmd-line argument.
   
     
\end{itemize}

